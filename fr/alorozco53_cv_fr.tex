%%%%%%%%%%%%%%%%%%%%%%%%%%%%%%%%%%%%%%%%%
\listfiles
\documentclass[11pt,a4paper,sans]{moderncv} %% Font sizes: 10, 11, or 12; paper sizes: a4paper, letterpaper, a5paper,
%%legalpaper, executivepaper or landscape; font families: sans or roman
\usepackage[utf8]{inputenc} %% codificación para la inserción acentos y otros carácteres especiales
\usepackage[T1]{fontenc} %% codificación para la impresión de acentos y otros carácteres especiales
\usepackage[french]{babel}
\graphicspath{{img/}}
\moderncvstyle{casual} %% CV theme - options include: 'casual' (default), 'classic', 'oldstyle' and 'banking'
\moderncvcolor{blue} %% CV color - options include: 'blue' (default), 'orange', 'green', 'red', 'purple', 'grey' and 'black'

%% \usepackage{lipsum} %% Used for inserting dummy 'Lorem ipsum' text into the template
\usepackage{multicol}
\AfterPreamble{\hypersetup{colorlinks=true}}
\usepackage[scale=0.75]{geometry} % Reduce document margins
%\setlength{\hintscolumnwidth}{3cm} % Uncomment to change the width of the dates column
%\setlength{\makecvtitlenamewidth}{10cm} % For the 'classic' style, uncomment to adjust the width of the space allocated to your name

%----------------------------------------------------------------------------------------
%	NAME AND CONTACT INFORMATION SECTION
%----------------------------------------------------------------------------------------

\firstname{Albert Manuel} % Your first name
\familyname{Orozco Camacho} % Your last name
% All information in this block is optional, comment out any lines you don't need
\title{Étudiant d'Informatique / Professeur Adjoint}
\address{356 Río Mixcoac Avenue, appt. 404}{03240 Mexico, Mexique}
\mobile{(+52 1) 55 3262 1338}
\email{alorozco.patriot53@gmail.com}
\photo[70pt][0.4pt]{picture} % The first bracket is the picture height, the second is the thickness of the frame around the picture (0pt for no frame)
\quote{"La science n'a pas de patrie, parce que le savoir est le patrimoine de l'humanité" -- Louis Pasteur}

%----------------------------------------------------------------------------------------

\begin{document}

\makecvtitle % Print the CV title

%----------------------------------------------------------------------------------------
%	EDUCATION SECTION
%----------------------------------------------------------------------------------------

\section{Cursus}

\cventry{2012--}{Licence (9e semestre)}{
  Facultad de Ciencias, Universidad Nacional Autónoma de México (UNAM)}{Mexico, Mexique}{\textit{Moyenne pondérée -- 8.86 / 10}
  \textit{Meilleur note: 9.54 / 10, Classement: 4e d'un génération de 150 étudiants}}{Informatique (à compléter bientôt)}
\cventry{2009--2012}{Lycée}{
  Prepa Tec de Monterrey (ITESM), Campus Guadalajara}{Guadalajara, Jalisco, Mexique}{\textit{Moyenne pondérée -- 91 / 100}
}{Mención honorífica, Bourse ITESM à $50\%$.}


%----------------------------------------------------------------------------------------
%	AWARDS SECTION
%----------------------------------------------------------------------------------------

%----------------------------------------------------------------------------------------
%	COMPUTER SKILLS SECTION
%----------------------------------------------------------------------------------------

\section{Compétences Informatiques}

\subsection{Langages de Programmation et Logiciels:}
\cvlistdoubleitem{Java}{Python}
\cvlistdoubleitem{C}{Haskell}
\cvlistdoubleitem{MatLab}{Prolog}
\cvlistdoubleitem{C++}{NumPy}
\cvlistdoubleitem{SciPy}{R}
\cvlistdoubleitem{Coq (assistant de preuve)}{PostgreSQL}

\renewcommand{\listitemsymbol}{-~}
\subsection{Logiciels complémentaires}
\cvlistdoubleitem{\LaTeX}{GNU Emacs}
\cvlistdoubleitem{Microsoft Office}{Tensorflow}
\cvlistdoubleitem{CMU Sphinx (reconnaisseur de parole)}{ScraPy}
\cvlistdoubleitem{Git (logiciel de gestion de versions)}{Theano}
\cvlistdoubleitem{Ionic Framework}{Javascript}
\newpage
\section{Compétences personnelles}
\cvlistitem{Capacité d'aborder des problèmes complexes.}
\cvlistitem{Capacité d'abstraire les caractéristiques les plus importantes d'une tâche donnée.}
\cvlistitem{Pensée critique.}
\cvlistitem{Capacité de travailler sous pression.}
\cvlistitem{Fournir des solutions de programmation efficaces pour tout problème.}

\section{Domaines d'intérêt}
\cvlistitem{Informatique}
\cvlistitem{Intelligence artificiel}
\cvlistitem{Apprentissage automatique}
\cvlistitem{Programation fonctionnelle}
\cvlistitem{Raisonnement automatisé}
\cvlistitem{Théorie des langages de programmation}
\cvlistitem{Théorie de la calculabilité}
\cvlistitem{Apprentissage profond}
\cvlistitem{Mathématiques discrètes}
\cvlistitem{Robotique}


\section{Experience}
\cventry{2016-}{Professeur adjoint à la Facultad de Ciencias (UNAM)}
        {Boulot payé}
        {J'ai enseigné le cours de \href{https://sites.google.com/site/lengprog20162/}{Langages de programmation} pendant le semestre de printemps 2016. Actuellement, j'enseigne le laboratoire du cours de \href{https://sites.google.com/a/ciencias.unam.mx/estructuras-discretas/home}{Structures discrètes} et \href{https://sites.google.com/site/logcompunam20171/home}{Logique computationnelle}. Tous les cours font partie du cursus d'informatique au niveau de premier cycle}
        {}{}{}
\cventry{2013-2015}{Étudiant / chercheur au \textit{Grupo Golem}, IIMAS, UNAM}
        {Activité parascolaire}
        {Grupo Golem est une groupe de recherche au IIMAS (Instituto de Investigaciones en Matemáticas Aplicadas y en Sistemas) dont le principal objectif est modéliser l'interaction cognitive entre un humaine et un ordinateur; toutes les recherches sont unifées dans un robot de service et mis en marche pour la compétition RoboCup@Home; le site  web du groupe est  \href{http://golem.iimas.unam.mx/home.php?lang=en&sec=home}{celui-ci}}
        {Grupo Golem's leader is \href{http://turing.iimas.unam.mx/~luis/}{Dr Luis A. Pineda}}{}{}
\cventry{2012-2013}{Étudiant /  chercheur au \textit{Departamento de Ciencias de la Computación}, IIMAS, UNAM}
        {Activité parascolaire}
        {J'ai travaillé avec le \href{http://turing.iimas.unam.mx/~ivanvladimir/}{Dr Ivan Vladimir Meza} dans le domaine de la reconnaissance vocal; puis, j'ai reproduit un expérience dans lequel un robot apprend d'un professeur humain à transformer des babillages en mots anglais simples}
        {}{}{}

\section{Other Activites / Workshops}

%% \cventry{2016}{}{}{}
        %% {I am currently working on my thesis project, which is a requirement for me to graduate at UNAM. With the advise of Dr. Ivan Vladimir Meza Ruiz, I will be building an automatic meme generator}{The paper's title is ``Random Forest Regression Experiments on Align-and-Differentiate and Word Embeddings penalizing strategies''. The paper can be visualized \href{http://aclweb.org/anthology/S/S16/S16-1112.pdf}{here}.}{}
%% {}

\cventry{2016}
        {Actuellement, je suis en train de travailler avec ma thèse de premier cycle}{Thèse}
        {Il s'agit de construire un générateur de memes d'Internet, en utilisant une quantité substantielle des images de memes etiqueteés afin d'entraîner une réseau neuronal profonde}
        {}{}

\cventry{2016}
        {J'ai participé dans le deuxième ``Jakatón'' (hackathon de linguistique computationnelle)}{Hackathon}
        {Il s'est tenue à Cholula, Puebla, Mexico}{}{}

\cventry{2015-2016}
        {J'ai fait partie d'une des tâches du compétence d'Evaluation Semantique (\href{http://alt.qcri.org/semeval2016/}{SemEval})}{Recherche}{J'ai travaillé avec des chercheurs dès le LIPN(Laboratoire d'Informatique du Paris Nord) et l'IIMAS-UNAM. On a publié un article qui présentait une solution pour un problème donnée.}
        {Le titre de l'article est ``Random Forest Regression Experiments on Align-and-Differentiate and Word Embeddings penalizing strategies''}
        {}

\cventry{2015}{J'ai assisté à l'\href{https://www.ee.washington.edu/news/2015JelinekWorkshopSummerSchool.html}{école d'été Jelinek sur les technologies du langage humain}}
        {École d'été}
        {Il s'agissait d'une progamme de 2 semaines d'introduction des sujets de pointe sur la reconnaissance vocale, l'apprentisage automatique, et le traitement du langage naturel}{Il s'est tenue à l'Université de Washington, Seattle, États-Unis}{J'ai reçu une bourse du North American Chapter of the Association for Computational Linguistics (NAACL) pour assister à l'école}

\cventry{2015}{J'ai fait partie du \href{http://naacl.org/reports/emerging_regions/Jakaton_report_apr2015.pdf}{premier hackathon de linguistique computationnelle au Mexique}}{Hackathon}{J'ai participé avec un ami et on a gagné une mention honorifique}{}{}

%----------------------------------------------------------------------------------------
%	LANGUAGES SECTION
%----------------------------------------------------------------------------------------

\section{Langages}

\cvitemwithcomment{Espagnol}{Langue maternelle}{}
\cvitemwithcomment{Anglais}{Compétence bilingue}{Complètement fluide}
\cvitemwithcomment{Français}{Capacité professionnelle complète}{Maîtrise la plupart du temps écrite}

%----------------------------------------------------------------------------------------
%	INTERESTS SECTION
%----------------------------------------------------------------------------------------

\section{Intérêts}

\renewcommand{\listitemsymbol}{-~} % Changes the symbol used for lists

\cvlistdoubleitem{Science fiction}{Robotique}
\cvlistdoubleitem{Football américain}{Football}
\cvlistdoubleitem{Science (en général)}{Faire des exercices}
\section{Détails Personnels}
\cvitem{Date de naissance}{16 juillet 1994}
\cvitem{Adresse permanent}{356 Río Mixcoac Avenue, appt. 404. 03240 Mexico, Mexique}
\cvitem{Langages}{Espagnol, Anglais, Français}
\cvitem{Numéro de portable}{(+52 1) 55 3262 1338}

\end{document}
