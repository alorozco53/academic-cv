%%%%%%%%%%%%%%%%%%%%%%%%%%%%%%%%%%%%%%%%%
\documentclass[11pt,a4paper,sans]{moderncv} %% Font sizes: 10, 11, or 12; paper sizes: a4paper, letterpaper, a5paper,
%%legalpaper, executivepaper or landscape; font families: sans or roman
\usepackage[utf8]{inputenc} %% codificación para la inserción acentos y otros carácteres especiales
\usepackage[T1]{fontenc} %% codificación para la impresión de acentos y otros carácteres especiales
\usepackage[english]{babel}
\graphicspath{{../images/}}
\moderncvstyle{casual} %% CV theme - options include: 'casual' (default), 'classic', 'oldstyle' and 'banking'
\moderncvcolor{blue} %% CV color - options include: 'blue' (default), 'orange', 'green', 'red', 'purple', 'grey' and 'black'

%% \usepackage{lipsum} %% Used for inserting dummy 'Lorem ipsum' text into the template
\usepackage{multicol}
\AfterPreamble{\hypersetup{colorlinks=true}}
\usepackage[scale=0.75]{geometry} % Reduce document margins
%\setlength{\hintscolumnwidth}{3cm} % Uncomment to change the width of the dates column
%\setlength{\makecvtitlenamewidth}{10cm} % For the 'classic' style, uncomment to adjust the width of the space allocated to your name

%----------------------------------------------------------------------------------------
%	NAME AND CONTACT INFORMATION SECTION
%----------------------------------------------------------------------------------------

\firstname{Albert Manuel} % Your first name
\familyname{Orozco Camacho} % Your last name
% All information in this block is optional, comment out any lines you don't need
\title{Estudiante / Ayudante de Asignatura}
\address{Av. Río Mixcoac 356 Int. 404 Del. Benito Juárez}{03240 Ciudad de México, CDMX, México}
\mobile{(+52 1) 55 3262 1338}
\email{alorozco.patriot53@gmail.com}
\photo[70pt][0.4pt]{picture} % The first bracket is the picture height, the second is the thickness of the frame around the picture (0pt for no frame)
\quote{"La science n'a pas de patrie, parce que le savoir est le patrimoine de l'humanité" -- Louis Pasteur}

%----------------------------------------------------------------------------------------

\begin{document}

\makecvtitle % Print the CV title

%----------------------------------------------------------------------------------------
%	EDUCATION SECTION
%----------------------------------------------------------------------------------------

\section{Educación}

\cventry{2012--}{Licenciatura}{
  Facultad de Ciencias, Universidad Nacional Aut\'{o}noma de M\'{e}xico (UNAM)}{Ciudad de México}{\textit{Promedio general -- 8.82 / 10}
}{Ciencias de la Computación (será concluído durante 2017)}
\cventry{2009--2012}{Bachillerato}{
  Prepa Tec de Monterrey, Campus Guadalajara}{Guadalajara, Jalisco, México}{\textit{Promedio general -- 91 / 100}
}{Título de bachillerato}


%----------------------------------------------------------------------------------------
%	AWARDS SECTION
%----------------------------------------------------------------------------------------

%----------------------------------------------------------------------------------------
%	COMPUTER SKILLS SECTION
%----------------------------------------------------------------------------------------

\section{Habilidades adicionales}

\subsection{Lenguajes de programación y software}
\begin{multicols}{3}
\cvitem {}
{\includegraphics[width=3cm,height=1.5cm]{java}}
\cvitem {}{\includegraphics[width=3cm,height=0.8cm]{python}}
\cvitem {}{\includegraphics[width=3cm,height=1.5cm]{c}}
\end{multicols}
\begin{multicols}{3}
\cvitem {}{\includegraphics[width=3cm,height=0.8cm]{haskell}}
\cvitem {}{\includegraphics[width=3cm,height=0.8cm]{matlab}}
\cvitem {}{\includegraphics[width=3cm,height=1.5cm]{swipl}}
\end{multicols}
\begin{multicols}{3}
\cvitem {}{\includegraphics[width=3cm,height=1.5cm]{cpp}}
\cvitem {}{\includegraphics[width=3cm,height=0.8cm]{numpy}}
\cvitem {}{\includegraphics[width=3cm,height=0.8cm]{scipy}}
\end{multicols}
\begin{multicols}{3}
\cvitem {}{\includegraphics[width=3cm,height=1.5cm]{r}}
\cvitem {}{\includegraphics[width=3cm,height=0.8cm]{coq}}
\cvitem {}{\includegraphics[width=3cm,height=0.8cm]{postgresql}}
\end{multicols}
\newpage
\renewcommand{\listitemsymbol}{-~}
\subsection{Software Adicional}
\cvlistdoubleitem{\LaTeX}{GNU Emacs}
\cvlistdoubleitem{Microsoft Office}{Tensorflow}
\cvlistdoubleitem{CMU Sphinx voice recognizer}{ScraPy}
\cvlistdoubleitem{Git version control system}{Theano}
\cvlistdoubleitem{Ionic Framework}{Javascript}
\section{Habilidades personales}
\cvlistitem{Habilidad para llevar a cabo problemas complejos.}
\cvlistitem{Habilidad para abstraer las características más importantes de una tarea dada.}
\cvlistitem{Pensamiento crítico.}
\cvlistitem{Habilidad para trabajar bajo presión.}
\cvlistitem{Habilidad para dar soluciones eficientes en problemas de programación..}

\section{Áreas de interés}
\cvlistitem{Ciencias de la computación}
\cvlistitem{Inteligencia artificial}
\cvlistitem{Aprendizaje automático}
\cvlistitem{Programación funcional}
\cvlistitem{Razonamiento automatizado}
\cvlistitem{Teoría de los lenguajes de programación}
\cvlistitem{Teoría de la computación}
\cvlistitem{Matemáticas discretas}
\cvlistitem{Robótica}


\section{Experiencia}
\cventry{2016-}{Ayudante de asignatura A (profesor adjunto) en la Facultad de Ciencias de la UNAM}
        {Trabajo pagado}
        {Actualmente enseño en el curso de \href{https://sites.google.com/site/automataslengformales20172/}{Autómatas y Lenguajes Formales}. Fui ayudante para el curso de \href{https://sites.google.com/site/lengprog20162/}{Lenguajes de Programación} durante el semestre de primavera de 2016, así como para el \href{https://sites.google.com/a/ciencias.unam.mx/estructuras-discretas/home}{laboratorio de Estructuras Discretas} y \href{https://sites.google.com/site/logcompunam20171/home}{Lógica Computacional} durante el semestre de otoño de 2016. Todos los cursos mencionados son parte del plan de estudios de Ciencias de la Computación}
        {}{}{}
\cventry{2013-2015}{Estudiante / investigador en el \textit{Grupo Golem} de la UNAM}
        {Actividad extracurricular}
        {El Grupo Golem es un grupo de investigación del IIMAS (Instituto de Investigaciones en Matem\'{a}ticas Aplicadas y en Sistemas) cuyo principal objetivo es modelar lainteracción cognitiva entre seres humanos y computadoras; toda la investigación se unifica en un robot de servicio que compite internacionalmente en la competencia Robocup@Home; la página web del grupo es \href{http://golem.iimas.unam.mx/home.php?lang=en&sec=home}{ésta}}
        {El líder del Grupo Golem es el \href{http://turing.iimas.unam.mx/~luis/}{Dr. Luis A. Pineda}}{}{}
\cventry{2012-2013}{Estudiante / investigador en el Departamento de Ciencias de la Computación del IIMAS, UNAM}
        {Actividad extracurricular}
        {Trabajé junto con el \href{http://turing.iimas.unam.mx/~ivanvladimir/}{Dr. Ivan V. Meza} con reconocedores de voz, y repliqué un experimento en el cual un robot aprende a pasar de balbuceos a palabras simples mediante la interacción con un maestro humano}
        {}{}{}

\section{Otras Actividades / Talleres}

\cventry{2017}
        {Voluntario en \href{http://technovationmx.org}{Technovation Challenge}}{Voluntariado}
        {El evento trata de fomentar la programación y emprendimiento a niñas de secundaria y preparatoria. Empecé a ayudar en la logística de una de las eliminatorias regionales durante el mes de mayo en la Ciudad de México}
        {}{}

\cventry{2016-2017}
        {Actualmente trabajo en mi tesis de licenciatura, la cual es un requisito para graduarme en la UNAM}{Tesis}
        {Estoy construyendo un generador automático de memes, utilizando una gran cantidad de imágenes obtenidas de Internet para entrenar una red neuronal profunda}
        {}{}

\cventry{2016}
        {Participé en el segundo ``Jakatón'' anual (hackathon de lingüística computacional)}{Hackathon}
        {Se llevó a cabo en Cholula, Puebla, México}{}{}

\cventry{2016}
        {Participé en una de las tareas del ``Semantic Evaluation'' (\href{http://alt.qcri.org/semeval2016/}{SemEval}) 2016}{Investigación}{Fui parte del equipo colaborativo entre el LIPN (Laboratoire d'Informatique du Paris Nord) y el IIMAS, en el cual se sometió un artículo que fue aceptado en esta ``competencia'' de lingüística computacional}
        {El título del artículo es ``Random Forest Regression Experiments on Align-and-Differentiate and Word Embeddings penalizing strategies''}
        {}

\cventry{2015}{Fui a la \href{https://www.ee.washington.edu/news/2015JelinekWorkshopSummerSchool.html}{Jelinek Summer School on Human Language Technologies}, edición 2015}
        {Escuela de verano}
        {Se trató de un seminario introductorio de dos semanas acerca de temas de vanguardia en los campos del reconocimietno del habla, aprendizaje automático y procesamiento de lenguaje natural}{Se llevó a cabo en la University of Washington, Seattle, EE.UU.}{Recibí una beca por parte de The North American Chapter of the Association for Computational Linguistics (NAACL) para ir a la escuela de verano.}

\cventry{2015}{Participé en el \href{http://naacl.org/reports/emerging_regions/Jakaton_report_apr2015.pdf}{primer hackathon mexicano de lingüística computacional}}{Hackathon}{Hice equipo con un amigo mío para llegar a ser finalistas y obtener una mención honorífica}{}{}

%----------------------------------------------------------------------------------------
%	LANGUAGES SECTION
%----------------------------------------------------------------------------------------

\section{Idiomas}

\cvitemwithcomment{Español}{Lengua materna}{}
\cvitemwithcomment{Inglés}{Competencia bilingüe}{Completamente fluído}
\cvitemwithcomment{Francés}{Competencia totalmente profesional}{Fluidez principalmente en la escritura y lectura}

%----------------------------------------------------------------------------------------
%	INTERESTS SECTION
%----------------------------------------------------------------------------------------

\section{Intereses}

\renewcommand{\listitemsymbol}{-~} % Changes the symbol used for lists

\cvlistdoubleitem{Ficción científica}{Robótica}
\cvlistdoubleitem{Fútbol asociación}{Fútbol americano}
\cvlistdoubleitem{Ciencia (en general)}{Hacer ejercicio}
\section{Datos personales}
\cvitem{Fecha de nacimiento}{16 de julio de 1994}
\cvitem{Dirección de casa}{Av. Río Mixcoac 356 Int. 404 Del. Benito Juárez. 03240 Ciudad de México, CDMX, México}
\cvitem{Idiomas}{Español, inglés y francés}
\cvitem{Teléfono móvil}{(+52 1) 55 3262 1338}

\end{document}
