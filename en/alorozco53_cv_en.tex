%%%%%%%%%%%%%%%%%%%%%%%%%%%%%%%%%%%%%%%%%
\documentclass[11pt,a4paper,sans]{moderncv} %% Font sizes: 10, 11, or 12; paper sizes: a4paper, letterpaper, a5paper,
%%legalpaper, executivepaper or landscape; font families: sans or roman
\usepackage[utf8]{inputenc} %% codificación para la inserción acentos y otros carácteres especiales
\usepackage[T1]{fontenc} %% codificación para la impresión de acentos y otros carácteres especiales
\usepackage[english]{babel}
\graphicspath{{../images/}}
\moderncvstyle{casual} %% CV theme - options include: 'casual' (default), 'classic', 'oldstyle' and 'banking'
\moderncvcolor{blue} %% CV color - options include: 'blue' (default), 'orange', 'green', 'red', 'purple', 'grey' and 'black'

%% \usepackage{lipsum} %% Used for inserting dummy 'Lorem ipsum' text into the template
\usepackage{multicol}
\AfterPreamble{\hypersetup{colorlinks=true}}
\usepackage[scale=0.75]{geometry} % Reduce document margins
%\setlength{\hintscolumnwidth}{3cm} % Uncomment to change the width of the dates column
%\setlength{\makecvtitlenamewidth}{10cm} % For the 'classic' style, uncomment to adjust the width of the space allocated to your name

%----------------------------------------------------------------------------------------
%	NAME AND CONTACT INFORMATION SECTION
%----------------------------------------------------------------------------------------

\firstname{Albert Manuel} % Your first name
\familyname{Orozco Camacho} % Your last name
% All information in this block is optional, comment out any lines you don't need
\title{Data Scientist / Teacher Assistant}
\address{356-404 Río Mixcoac Ave.}{03240 Mexico City, Mexico}
\mobile{(+52 1) 55 3262 1338}
\email{alorozco53@ciencias.unam.mx}
\photo[70pt][0.4pt]{picture} % The first bracket is the picture height, the second is the thickness of the frame around the picture (0pt for no frame)
\quote{"La science n'a pas de patrie, parce que le savoir est le patrimoine de l'humanité" -- Louis Pasteur}

%----------------------------------------------------------------------------------------

\begin{document}

\makecvtitle % Print the CV title

%----------------------------------------------------------------------------------------
%	EDUCATION SECTION
%----------------------------------------------------------------------------------------

\section{Education}

\cventry{2012-2017}{Bachelor in Science Degree}{
  Facultad de Ciencias, Universidad Nacional Aut\'{o}noma de M\'{e}xico (UNAM)}{Mexico City}{\textit{GPA -- 8.82 / 10}
}{Computer Science}
\cventry{2009--2012}{High School}{
  Prepa Tec de Monterrey, Campus Guadalajara}{Guadalajara, Jalisco, Mexico}{\textit{GPA -- 91 / 100}
}{Secondary School Certificate}


%----------------------------------------------------------------------------------------
%	AWARDS SECTION
%----------------------------------------------------------------------------------------

%----------------------------------------------------------------------------------------
%	COMPUTER SKILLS SECTION
%----------------------------------------------------------------------------------------

\section{Skills}

\renewcommand{\listitemsymbol}{-~}
\subsection{Programming Languages and Software known:}
\cvlistdoubleitem{Java}{Python}
\cvlistdoubleitem{C}{Haskell}
\cvlistdoubleitem{MatLab}{Prolog}
\cvlistdoubleitem{C++}{NumPy}
\cvlistdoubleitem{SciPy}{R}
\cvlistdoubleitem{Sci-kit Learn}{spaCy}
\cvlistdoubleitem{NLTK}{Matplotlib}
\cvlistdoubleitem{OpenCV}{PostgreSQL}

\subsection{Additional Software}
\cvlistdoubleitem{\LaTeX}{GNU Emacs}
\cvlistdoubleitem{Microsoft Office}{Tensorflow}
\cvlistdoubleitem{CMU Sphinx voice recognizer}{ScraPy}
\cvlistdoubleitem{Git version control system}{Theano}
\cvlistdoubleitem{IBM Watson}{Javascript}
\cvlistdoubleitem{Keras}{Bash (Shell)}
\cvlistdoubleitem{Amazon AWS}{Google Cloud}

\section{Personal Skills}
\cvlistitem{Ability to tackle complex problems.}
\cvlistitem{Ability to abstract the most important features of a given task.}
\cvlistitem{Critical thinking.}
\cvlistitem{Ability to work under pressure}
\cvlistitem{Provide efficient programming solutions to any problem.}

\section{Areas of Interest}
\cvlistitem{Computer Science}
\cvlistitem{Artificial Intelligence}
\cvlistitem{Machine Learning}
\cvlistitem{Data Science}
\cvlistitem{Functional Programming}
\cvlistitem{Automated Reasoning}
\cvlistitem{Programming Language Theory}
\cvlistitem{Theory of Computation}
\cvlistitem{Robotics}


\section{Experience}

\cventry{2017-}{Data Scientist at \href{http://www.mariachi.io}{Mariachi IO}}
        {Paid Job}
        {I joined \emph{Mariachi IO} to help out in the solution of several tasks that require NLP, machine learning, and image processing}
        {I am currently building a computer vision application using classic OpenCV-based algorithms and state-of-the art deep learning tools}
        {}
        {}

\cventry{2017-}{Data Science Consultant at \href{http://www.fractalabogados.com}{Fractal Abogados}}
        {Paid Job}
        {\emph{Fractal Abogados} is a startup whose purpose is to provide feasible IT (and AI) solutions into today's Mexican (and Latin American) law system}
        {I work as an AI consultant in Fractal's signature project: a legal chatbot. \href{https://m.me/fractal-abogados}{\emph{Max}} is a Facebook Messenger based virtual assistant, powered by IBM Watson that automates the most common legal advices in Mexico}
        {}
        {}

\cventry{2016-}{Teacher assistant the UNAM's Facultad de Ciencias}
        {Paid Job}
        {Taught \href{http://turing.iimas.unam.mx/~ivanvladimir/page/curso_rpyaa}{Machine Learning and Pattern Recognition} and \href{https://sites.google.com/site/automataslengformales20172/}{Automata and Formal Languages} during 2017. Taught the \href{https://sites.google.com/site/lengprog20162/}{Programming Languages} during the 2016 Spring Semester. Taught \href{https://sites.google.com/a/ciencias.unam.mx/estructuras-discretas/home}{Discrete Structures Lab} and \href{https://sites.google.com/site/logcompunam20171/home}{Computational Logic} during the 2016 Fall Semester. All courses are offered for the undergraduate curriculum}
        {I will be teaching \emph{Computational Logic} again this Spring 2018 semester}{}{}
\cventry{2013-2015}{Student / researcher at UNAM's \textit{Grupo Golem}}
        {Extra-curricular activity}
        {Grupo Golem is a research group at IIMAS (Instituto de Investigaciones en Matem\'{a}ticas Aplicadas y en Sistemas) whose main goal is to model the cognitive interaction between a humans and computer; all the research is unified in a service robot that competes internationally in the RoboCup@Home competition; the group's website is \href{http://golem.iimas.unam.mx/home.php?lang=en&sec=home}{this one}}
        {Grupo Golem's leader is \href{http://turing.iimas.unam.mx/~luis/}{Dr Luis A. Pineda}}{}{}
\cventry{2012-2013}{Student / researcher at UNAM's IIMAS's Computer Science Department}
        {Extra-curricular activity}
        {I work alongside \href{http://turing.iimas.unam.mx/~ivanvladimir/}{Dr Ivan V. Meza} with speeech recognizers, and replicated an experiment in which a robot learns from a human teacher how to transform babblings to simple English words}
        {}{}{}

\section{Publications}

\cventry{2016}
        {Participated in one the 2016 Semantic Evaluation (\href{http://alt.qcri.org/semeval2016/}{SemEval}) tasks}{Research}{I was part of the joint LIPN(Laboratoire d'Informatique du Paris Nord)-IIMAS team to submit a paper that was accepted in this computational linguistics ``competition''}
        {The paper's title is ``Random Forest Regression Experiments on Align-and-Differentiate and Word Embeddings penalizing strategies''}
        {\url{https://www.aclweb.org/anthology/S/S16/S16-1112.pdf}}

\cventry{2014}
        {While being part of the \emph{Golem group}, a paper was published in order to present that year's work to the community}{Research}
        {The paper's title is ``The Golem Team, RoboCup@Home 2014''}{}
        {\url{http://golem.iimas.unam.mx/pubs/tdp_Golem-II+_2014.pdf}}

\section{Talks}

\cventry{2017}
        {Talk delivered at \href{https://www.meetup.com/GDG-UNAM/}{UNAM Google Developer Group}'s meetup and \href{http://www.escom.ipn.mx}{IPN ESCOM}}
        {Outreach talk}
        {I spoke about my undergraduate thesis project and the lessons I learned during the coding part}{}
        {The slides are available in \url{https://alorozco53.github.io/talks/lessons.html}}

\cventry{2017}
        {Talk delivered at \href{http://www.corpus.unam.mx/colico/VIIICoLiCo.html}{CoLiCo} held at UNAM Facultad de Filosofía y Letras}
        {Outreach talk}
        {I spoke about deep learning applications to NLP, from a linguistic perspective}
        {\emph{CoLiCo} stands for ``Computational Linguistics Colloquium'' and was organized by UNAM Linguistic Engineering Group}
        {The slides are available in \url{https://alorozco53.github.io/talks/onto_memes.html}}

\cventry{2017}
        {In March, I spoke at the ``Bots LATAM'' community, whose goal is to gather the most enthusiastic people in AI and Chatbots together in Mexico City}
        {Outreach talk}
        {The talk was given in Spanish and its title was ``Implementando ojos a tu chatbot''}{}
        {The slides are available in \url{https://alorozco53.github.io/talks/eyes_on_bot.html}}

\section{Other Activites / Workshops}

\cventry{2017}
        {Poster speaker at SOCML}
        {Conference presentation}
        {I presented a poster at the second Self-Organizing Conference on Machine Learning, held at the Google offices located in Sunnyvale, CA, USA. The poster was about my undergraduate thesis project (deep meme captioning). The conference was organized by Ian Goodfellow}
        {The poster can be visualized \href{https://drive.google.com/file/d/1DPjqiXcmliwPZBOfBIXxnIX_rs9EPUub/view?usp=sharing}{here}}
        {}

\cventry{2017}
        {Attended a NLP hackathon (\emph{Gilkatón}) organized by UNAM's
          \href{http://grupos.iingen.unam.mx/iling/es-mx/Paginas/default.aspx}{Grupo de Ingeniería Lingüística}}
        {Hackathon}
        {Held in Mexico City at the Engineering Tower, UNAM}
        {Alongside two linguists, we created a program to extract relevant information
          from a corpus of legal documents}
        {The code developed can be found here: \url{https://github.com/alorozco53/Gilkaton}.}

\cventry{2017}
        {Attended an AI-Chatbot Hackathon organized by \href{http://synx.co}{Synx} and \href{https://www.recime.io}{Recime}}
        {Hackathon}
        {Held in Tlaquepaque, Jalisco, Mexico at \href{http://www.iteso.mx}{ITESO}}{}
        {Alongside three friends, I helped developing a NLP module for a chatbot that tracks the user's food quality and exercising activites.}

\cventry{2017}
        {Volunteer in \href{http://technovationmx.org}{Technovation Challenge}}{Volunteering}
        {This event's focus is to encourage programming and entrepreneurship in Secondary School girls. During one of the regional knockout events, I helped out with logistic issues; it was held during May at Mexico City}
        {}{}

\cventry{2016}
        {Currently working on my thesis project, which is a requirement for me to graduate at UNAM}{Thesis}
        {I am building an automatic meme generator, using a substantial quantity of meme images and captions to train a deep neural network. I am using a convolutional neural network and a recurrent network to encode meme images and captions into two vector spaces. Furthermore, by connecting this architecture, I am able to train it using data gathered from the Web in order to predict the most likely sequence of words to label an input image}
        {}{}

\cventry{2016}
        {Participated in the second annual ``Jakatón'' (computational linguistics hackathon)}{Hackathon}
        {Held in Cholula, Puebla, Mexico}{}{}

\cventry{2015}{Attended the 2015 \href{https://www2.ee.washington.edu/news/2015JelinekWorkshopSummerSchool.html}{Jelinek Summer School on Human Language Technologies}}
        {Summer school}
        {It was a 2-week introductory summer school on cutting-edge topics about speech recognition, machine learning, and natural language processing}{It was held at the University of Washington, Seattle}{Received a scholarship from The North American Chapter of the Association for Computational Linguistics (NAACL) to attend the summer school.}

\cventry{2015}{Participated in \href{http://naacl.org/reports/emerging_regions/Jakaton_report_apr2015.pdf}{Mexico's first computational linguistics hackathon}}{Hackathon}{I teamed up with a friend of mine to be finalists and obtained an honor's award}{}{}

%----------------------------------------------------------------------------------------
%	LANGUAGES SECTION
%----------------------------------------------------------------------------------------

\section{Languages}

\cvitemwithcomment{Spanish}{Mothertongue}{}
\cvitemwithcomment{English}{Bilingual proficiency}{Completely fluent (B2)}
\cvitemwithcomment{French}{Full professional proficiency}{Fluency mostly written (DELF B1)}

%----------------------------------------------------------------------------------------
%	INTERESTS SECTION
%----------------------------------------------------------------------------------------

\section{Interests}

\renewcommand{\listitemsymbol}{-~} % Changes the symbol used for lists

\cvlistdoubleitem{Science-fiction}{Robotics}
\cvlistdoubleitem{Soccer}{American football}
\cvlistdoubleitem{Science (in general)}{Exercising}
\newpage
\section{Personal Details}
\cvitem{Date of birth}{July 16th, 1994}
\cvitem{Website}{\url{https://alorozco53.github.io/}}
\cvitem{GitHub page}{\url{https://github.com/alorozco53}}
\cvitem{Linkedin page}{\url{https://www.linkedin.com/in/albert-orozco-camacho-ba315b3a/}}
\cvitem{Languages}{Spanish, English, French}
\cvitem{Permanent address}{356-404 Río Mixcoac Ave. 03240 Mexico City, Mexico}
\cvitem{Mobile No.}{(+52 1) 55 3262 1338}

\end{document}
